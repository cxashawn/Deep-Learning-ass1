%%%%%%%%%%%%%%%%%%%%%%%%%%%%%%%%%%%%%%%%%
% Short Sectioned Assignment
% LaTeX Template
% Version 1.0 (5/5/12)
%
% This template has been downloaded from:
% http://www.LaTeXTemplates.com
%
% Original author:
% Frits Wenneker (http://www.howtotex.com)
%
% License:
% CC BY-NC-SA 3.0 (http://creativecommons.org/licenses/by-nc-sa/3.0/)
%
%%%%%%%%%%%%%%%%%%%%%%%%%%%%%%%%%%%%%%%%%

%----------------------------------------------------------------------------------------
%	PACKAGES AND OTHER DOCUMENT CONFIGURATIONS
%----------------------------------------------------------------------------------------

\documentclass[paper=a4, fontsize=11pt]{scrartcl} % A4 paper and 11pt font size

\usepackage[T1]{fontenc} % Use 8-bit encoding that has 256 glyphs
\usepackage{fourier} % Use the Adobe Utopia font for the document - comment this line to return to the LaTeX default
\usepackage[english]{babel} % English language/hyphenation
\usepackage{amsmath,amsfonts,amsthm} % Math packages

\usepackage{lipsum} % Used for inserting dummy 'Lorem ipsum' text into the template

\usepackage{sectsty} % Allows customizing section commands
\allsectionsfont{\raggedright \normalfont\scshape} % Make all sections centered, the default font and small caps

\usepackage{fancyhdr} % Custom headers and footers
\pagestyle{fancyplain} % Makes all pages in the document conform to the custom headers and footers
\fancyhead{} % No page header - if you want one, create it in the same way as the footers below
\fancyfoot[L]{} % Empty left footer
\fancyfoot[C]{} % Empty center footer
\fancyfoot[R]{\thepage} % Page numbering for right footer
\renewcommand{\headrulewidth}{0pt} % Remove header underlines
\renewcommand{\footrulewidth}{0pt} % Remove footer underlines
\setlength{\headheight}{13.6pt} % Customize the height of the header

\numberwithin{equation}{section} % Number equations within sections (i.e. 1.1, 1.2, 2.1, 2.2 instead of 1, 2, 3, 4)
\numberwithin{figure}{section} % Number figures within sections (i.e. 1.1, 1.2, 2.1, 2.2 instead of 1, 2, 3, 4)
\numberwithin{table}{section} % Number tables within sections (i.e. 1.1, 1.2, 2.1, 2.2 instead of 1, 2, 3, 4)

\setlength\parindent{0pt} % Removes all indentation from paragraphs - comment this line for an assignment with lots of text


%----------------------------------------------------------------------------------------
%	TITLE SECTION
%----------------------------------------------------------------------------------------

\newcommand{\horrule}[1]{\rule{\linewidth}{#1}} % Create horizontal rule command with 1 argument of height

\title{	
\normalfont \normalsize 
\textsc{Courant Institute of Mathematics Sciences, New York University} \\ [25pt] % Your university, school and/or department name(s)
\horrule{0.5pt} \\[0.4cm] % Thin top horizontal rule
\huge Deep Learning Assignment 1 \\ % The assignment title
\horrule{2pt} \\[0.5cm] % Thick bottom horizontal rule
}

\author{Wenliang Zhao \& Xuanang Chen} % Your name

\date{\normalsize\today} % Today's date or a custom date

\begin{document}

\maketitle % Print the title



%----------------------------------------------------------------------------------------
%	PROBLEM 1
%----------------------------------------------------------------------------------------

\section{Problem 1}

%\lipsum[2] % Dummy text

%------------------------------------------------

\subsection{}
Let $f(x_{in}) = \frac{1}{1 + \exp^{-x_{in}}}$\\
\begin{align} 
\begin{split}
\frac{\partial E}{\partial x_{in}} &= \frac{\partial E}{\partial f(x_{in})} \frac{\partial f(x_{in})}{x_{in}}\\
&=\frac{\partial E}{\partial x_{out}} \frac{\partial f(x_{in})}{\partial x_{in}}\\
\frac{\partial f(x_{in})}{\partial x_{in}} &= \frac{\partial}{\partial x_{in}} \frac{1}{1 + \exp^{-x_{in}}}\\
&=\frac{\exp^{-x_{in}}}{(1 + \exp^{-x_{in}})^2}\\
\Longrightarrow & \frac{\partial E}{\partial x_{in}} = \frac{\partial E}{\partial x_{out}} \frac{\exp^{-x_{in}}}{(1 + \exp^{-x_{in}})^2}
\end{split}         
\end{align}



%------------------------------------------------

\subsection{}

%\lipsum[3] % Dummy text

%\paragraph{Heading on level 4 (paragraph)}
For $i = j$\\
\begin{align} 
\begin{split}
\frac{\partial (x_{out})_i}{\partial (x_{in})_j} &= \frac{\partial}{\partial (x_{in})_i} \frac{\exp^{-\beta(x_{in})_i}}{\sum_k{\exp^{-\beta(x_{in})_k}}}\\ 
&= (\frac{\partial}{\partial (x_{in})_i} \exp^{-\beta(x_{in})_k}) \times \frac{1}{\sum_k{\exp^{-\beta(x_{in})_k}}} +  (\frac{\partial}{\partial (x_{in})_i} \frac{1}{\sum_k{\exp^{-\beta(x_{in})_k}}}) \times \exp^{-\beta(x_{in})_i}\\
&= (-\beta)\exp^{-\beta{(x_{in})_i}} \frac{1}{\sum_k{\exp^{-\beta(x_{in})_k}}} - \exp^{-\beta(x_{in})_i} \frac{1}{(\sum_k{\exp^{-\beta(x_{in})_k}})^2} (-\beta) \exp^{-\beta(x_{in})_i}\\ 
&= -\beta \frac{\exp^{-\beta(x_{in})_i}}{\sum_k{\exp^{-\beta(x_{in})_k}}} (1 - \frac{\exp^{-\beta(x_{in})_i}}{\sum_k{\exp^{-\beta(x_{in})_k}}})
\end{split}         
\end{align}

%\lipsum[6] % Dummy text

%----------------------------------------------------------------------------------------
%	PROBLEM 2
%----------------------------------------------------------------------------------------
\section{Torch (MNIST Handwritten Digit Recognition)}

%----------------------------------------------------------------------------------------

\end{document}
